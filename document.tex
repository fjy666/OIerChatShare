\documentclass{beamer}
\usepackage{ctex}
\usepackage{hyperref}
\title{Suffix Array}
\author{WAKing}
\institute{OIers'Chat}
\date{11th June, 2022}
\usetheme{Copenhagen}
\usecolortheme{seahorse}
\AtBeginSection[]{
    \begin{frame}
        \frametitle{Contents}
        \tableofcontents[currentsection]
    \end{frame}
}
\begin{document}
    \frame{\titlepage}
    \section{什么是后缀数组?}
    \begin{frame}
        \frametitle{什么是后缀数组}
        \pause
        后缀数组就是把一个字符串所有的后缀按照字典序排序后排名为第 $i$ 的后缀开头的位置形成的数组\\
        \pause
        例如 $\text{"abb"}$ 的后缀数组就是 $\left[1,3,2\right]$
    \end{frame}
    \section{怎么求后缀数组?}
    \begin{frame}
        \frametitle{怎么求后缀数组}
    \end{frame}
    \subsection{$O(n\log^2n)$ 的垃圾哈希做法}
    \begin{frame}
        \frametitle{哈希}
        \pause
        注意到我们可以利用哈希 $+$ 二分在 $O(\log n)$ 的复杂度内求出两个串的最长公共前缀 $\operatorname{lcp}$,然后比较后一位就可以判断出来两个串的大小\\
        \pause
        所以我们考虑用这个方式比较,然后用归并排序什么的就可以做到 $O(n\log^2n)$ 求出后缀数组了
    \end{frame}
    \subsection{$O(n\log n)$ 的倍增做法}
    \begin{frame}
        \frametitle{倍增}
        \pause
        注意到如果我们要比较两个字符串 $a$ 和 $b$,可以先比较 $a$ 的前 $k$ 位和 $b$ 的前 $k$ 位,如果不一样就可以直接确定,一样的话再比较后面的字符串\\
        \pause
        于是我们考虑倍增,首先先按照每一位的字符排个序,然后接下来进行 $\log n$ 轮排序,第 $i$ 轮就是按照每个后缀的前 $2^i$ 位排序,如果长度不到 $2^i$,后面用空字符替代\\
    \end{frame}
    \begin{frame}
        \frametitle{倍增}
        \pause
        具体的,我们在第 $k+1$ 轮比较两个开始位置是 $a$ 和 $b$ 的后缀时,可以先比较在上一轮时的排名,如果相等,再去比较 $a+2^k$ 和 $b+2^k$ 的后缀在上一轮的排名\\
        \pause
        实际上容易发现我们第 $k+1$ 轮排序时就是以后缀 $i$ 在上一轮的排名为第一关键字、后缀 $i+2^k$ 在上一轮的排名(如果 $i+2^k$ 大于 $n$ 的话排名就是 $0$ )为第二关键字排序\\
        \pause
        可以用基数排序做,时间复杂度 $O(n\log n)$
    \end{frame}
    \subsubsection{实现}
    \begin{frame}
        \frametitle{实现}
        \pause
        \href{https://www.luogu.com.cn/paste/1tdjoxie}{\text{code}}
    \end{frame}
    \section{求了后缀数组有什么用?}
    \begin{frame}
        \frametitle{求了后缀数组有什么用?}
    \end{frame}
    \subsection{Height 数组}
    \begin{frame}
        \frametitle{Height 数组}
        \pause
        Height 数组就是排名相邻的两个后缀的最长公共前缀,也就是 $\operatorname{Height}_i=\operatorname{lcp}\left(s\left[rk_{i-1},rk_{i-1}+1,rk_{i-1}+2,\cdots,n\right],s\left[rk_i,rk_i+1,rk_i+2,\cdots,n\right]\right)$\\
        \pause
        特别的 $\operatorname{Height}_1=0$\\
        \pause
        定理:$\operatorname{Height}_{\operatorname{rk}_{i-1}}-1\leq\operatorname{Height}_{\operatorname{rk}_i}$\\
        \pause
        证明懒得写了,见 oi-wiki\\
        \pause
        利用上面的定理,直接暴力去做就可以做到线性求出 $\operatorname{Height}$ 数组
    \end{frame}
    \subsection{任意两个后缀的 $\operatorname{lcp}$}
    \begin{frame}
        \frametitle{任意两个后缀的 $\operatorname{lcp}$}
        \pause
        容易注意到 $\operatorname{lcp}\left(s\left[a,a+1,a+2,\cdots,n\right],s\left[b,b+1,b+2,\cdots,n\right]\right)=\min\limits_{k=\operatorname{rk}_a+1}^{\operatorname{rk}_b}\operatorname{Height}_k$\\
        \pause
        所以可以转成 RMQ 问题,用四毛子算法解决就行了
    \end{frame}
\end{document}